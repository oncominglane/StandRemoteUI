\documentclass[12pt]{article}
\usepackage[utf8]{inputenc}
\usepackage[russian]{babel}
\usepackage{geometry}
\usepackage{titlesec}
\usepackage{enumitem}
\usepackage{fancyhdr}
\usepackage{hyperref}
\usepackage{datetime}
\geometry{a4paper, margin=2.5cm}
\titleformat{\section}{\large\bfseries}{\thesection.}{1em}{}
\pagestyle{fancy}
\fancyhf{}
\lhead{Инструкция пользователя}
\rhead{\today}
\cfoot{\thepage}

\begin{document}

\begin{center}
    \LARGE\textbf{Руководство пользователя приложения StandControl} \\
    \large Версия: 25.12.2025
\end{center}

\section{Назначение}

Программа StandControl предназначена для удалённого управления и мониторинга инвертора электропривода на стенде испытаний. Приложение обеспечивает задание режимов управления, передачу команд по CAN-шине, приём и отображение телеметрии в реальном времени, ведение журнала измерений, построение трендов и карт параметров электродвигателя.

Архитектура системы клиент--серверная. Клиентское приложение предоставляет графический интерфейс пользователя, серверная часть осуществляет обмен с инвертором по CAN и ретрансляцию данных клиенту.

\section{Порядок работы}

\subsection*{1. Запуск и подключение}

\begin{itemize}[leftmargin=1.5em]
    \item Запустите серверную часть StandControl Server на компьютере, подключённом к инвертору по CAN.
    \item Запустите клиентское приложение StandControl GUI на устройстве, находящемся в одной локальной сети с первым.
    \item При успешном подключении индикатор соединения в верхней панели изменит цвет на зелёный.
\end{itemize}

\subsection*{2. Выбор режима управления}

Доступны 3 режима управления электроприводом:
\begin{itemize}[leftmargin=1.5em]
    \item \textbf{Currents (Id/Iq)} — управление по токам статора в $dq$-координатах.
    \item \textbf{Frequency (ns)} — управление по частоте вращения (обороты).
    \item \textbf{Torque (Ms)} — управление по моменту.
\end{itemize}

Выбор режима осуществляется в блоке \textbf{Control mode}. После изменения режима необходимо нажать кнопку \textbf{Отправить} или \textbf{Применить режим}.

\subsection*{3. Выбор передачи}

В блоке \textbf{Gear (D/R/N)} задаётся передача:
\begin{itemize}[leftmargin=1.5em]
    \item \textbf{D} — движение вперёд;
    \item \textbf{R} — реверс;
    \item \textbf{N} — нейтраль.
\end{itemize}

Выбранное значение передаётся на сервер немедленно.

\subsection*{4. Задание управляющих параметров}

\textbf{Режим Currents (Id/Iq):}
\begin{itemize}[leftmargin=1.5em]
    \item Введите значения \textbf{Id} и \textbf{Iq} (А) вручную или с помощью ползунков.
    \item Нажмите \textbf{SendTorque (Id/Iq)} или \textbf{Отправить}.
\end{itemize}

\textbf{Режим Frequency (ns):}
\begin{itemize}[leftmargin=1.5em]
    \item Задайте требуемую частоту вращения (rpm).
    \item Нажмите \textbf{Отправить}.
\end{itemize}

\subsection*{5. Установка ограничений}

В блоке \textbf{Limits} задаются:
\begin{itemize}[leftmargin=1.5em]
    \item минимальный и максимальный момент (\textbf{M\_min}, \textbf{M\_max});
    \item максимальный градиент момента (\textbf{M\_grad\_max});
    \item максимальная скорость вращения (\textbf{n\_max}).
\end{itemize}

Для передачи ограничений нажмите кнопку \textbf{SendLimits}.

\subsection*{6. Запуск и останов}

\begin{itemize}[leftmargin=1.5em]
    \item \textbf{Start} — инициализация и перевод системы в рабочее состояние.
    \item \textbf{Stop} — останов управления.
    \item \textbf{Reset} — сброс состояния и повторное чтение параметров.
    \item \textbf{Save} — сохранение текущей конфигурации (если поддерживается сервером).
\end{itemize}

\section{Индикация и телеметрия}

Во вкладке \textbf{Indication} отображаются основные параметры:
\begin{itemize}[leftmargin=1.5em]
    \item скорость вращения;
    \item момент;
    \item токи и напряжения в $dq$-осях;
    \item температуры и дополнительные параметры MCU/VCU (при наличии).
\end{itemize}

В разделе \textbf{Tx / Rx CAN} отображаются последние переданные и принятые CAN-кадры.

\section{Журнал и экспорт данных}

Во вкладке \textbf{Logbook} ведётся журнал телеметрии:
\begin{itemize}[leftmargin=1.5em]
    \item каждая строка содержит временную метку и набор измеренных параметров;
    \item журнал можно очистить кнопкой \textbf{Clear};
    \item данные экспортируются в CSV кнопкой \textbf{Export CSV}.
\end{itemize}

В нижней части вкладки отображается текстовый лог событий.

\section{Тренды и карты}

\subsection*{Тренды}

Во вкладке \textbf{Trends} в реальном времени строятся графики:
\begin{itemize}[leftmargin=1.5em]
    \item скорость вращения;
    \item момент;
    \item токи и напряжения;
    \item дополнительные параметры электропривода.
\end{itemize}

Графики автоматически масштабируются по времени и амплитуде.

\subsection*{Карты}

Во вкладке \textbf{Maps} отображаются:
\begin{itemize}[leftmargin=1.5em]
    \item зависимости \textbf{Ld(Id)} и \textbf{Lq(Iq)}, вычисляемые онлайн;
    \item карты \textbf{момент / мощность – обороты}.
\end{itemize}

Данные накапливаются по мере поступления телеметрии.

\section{Замечания}

\begin{itemize}[leftmargin=1.5em]
    \item При отсутствии связи с сервером управление блокируется.
    \item Изменения параметров не передаются автоматически — требуется нажатие кнопки \textbf{Отправить}.
    \item Для корректной работы требуется установленный драйвер CAN-интерфейса.
\end{itemize}

\vspace{2em}
\hrule
\vspace{1em}
\hrule
\vspace{1em}
\begin{center}
    \textbf{Контакты} \\[0.5em]
    Маков Сергей Константинович \\[0.5em]
    Email: \href{mailto:makov@amtc.org}{\texttt{makov@amtc.org}} \\[0.3em]
    Telegram: \href{https://t.me/TopSerg}{\texttt{@TopSerg}} \\[0.3em]

\end{center}

\end{document}
